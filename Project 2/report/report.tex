\documentclass[12pt,a4paper]{article}
\usepackage[utf8]{inputenc}
\usepackage[T1]{fontenc}
\usepackage{amsmath,amsfonts,amssymb,amsthm}
\usepackage{geometry}
\usepackage{graphicx}
\usepackage{float}
\usepackage{hyperref}
\usepackage{booktabs}
\usepackage{multirow}
\usepackage{array}
\usepackage{xcolor}
\usepackage{subcaption}
\usepackage{cite}
\usepackage{siunitx}
\usepackage{enumitem}

\geometry{margin=1in}

% Custom commands
\newcommand{\E}{\mathbb{E}}
\newcommand{\Var}{\text{Var}}
\newcommand{\Cov}{\text{Cov}}
\newcommand{\R}{\mathbb{R}}

% Title Page
\title{\textbf{Strategic Asset Allocation for MySuper Product: \\
Portfolio Construction and Optimization using Market Data and APRA Guidelines}}
\author{SID: 530522058 \\ FMAT3888 Projects in Financial Mathematics \\ Group 10}
\date{\today}

\begin{document}

\maketitle
\thispagestyle{empty}
\newpage

\tableofcontents
\newpage

% ==========================================
% EXECUTIVE SUMMARY
% ==========================================
\section{Executive Summary}

This report presents a comprehensive Strategic Asset Allocation (SAA) model for a MySuper balanced fund, targeting 70\% growth and 30\% defensive assets. Using Markowitz Modern Portfolio Theory and advanced optimization techniques, we developed a portfolio that enhances risk-adjusted returns while maintaining close alignment with industry benchmarks.

Our optimized portfolio achieves a Sharpe ratio of 0.862, improving upon the benchmark's 0.809, while maintaining a tracking error of only 1.31\%. The portfolio meets the target return requirement of 5.594\% (CPI + 3\%) with an expected return of 8.34\% and reduced volatility of 7.36\% compared to the benchmark's 8.20\%.

Key strategic tilts include overweighting international equities and infrastructure for growth exposure while increasing cash holdings for liquidity management. The implementation cost is estimated at 7 basis points with ongoing management expense increases of 1.7 basis points.

% ==========================================
% INTRODUCTION
% ==========================================
\section{Introduction}

\subsection{Background}
The Australian superannuation industry manages over \$3.5 trillion in assets, with MySuper products serving as the default option for millions of members. Strategic Asset Allocation decisions are critical for these products, as they must balance long-term growth objectives with appropriate risk management while adhering to APRA guidelines \cite{apra2025}.

Portfolio optimization has evolved significantly since Markowitz's seminal work \cite{markowitz1952}, with modern approaches incorporating parameter uncertainty, non-normal distributions, and dynamic rebalancing strategies. This project applies both classical and contemporary techniques to construct an optimal SAA for a MySuper balanced fund.

\subsection{Objectives}
This report addresses the following objectives:
\begin{enumerate}
    \item Estimate expected returns and covariances using historical data with consideration for estimation error
    \item Construct efficient portfolios subject to regulatory and practical constraints
    \item Compare static and dynamic allocation strategies
    \item Implement advanced risk management techniques including non-positive semidefinite covariance matrix handling
    \item Evaluate utility-based optimization under log-normal asset dynamics
\end{enumerate}

\subsection{Report Structure}
Section 3 presents the mathematical framework for portfolio optimization. Section 4 details our parameter estimation methodology. Section 5 presents the theoretical derivations. Section 6 discusses computational results and portfolio recommendations. Section 7 concludes with implementation considerations.

% ==========================================
% MATHEMATICAL SETUP
% ==========================================
\section{Mathematical Setup}

\subsection{Portfolio Optimization Framework}

Consider a universe of $n$ assets with random returns $\mathbf{R} = (R_1, ..., R_n)^T$. The portfolio optimization problem seeks weight vector $\mathbf{w} \in \R^n$ that solves:

\begin{align}
\min_{\mathbf{w}} \quad & \mathbf{w}^T \Sigma \mathbf{w} \\
\text{subject to} \quad & \mathbf{w}^T \boldsymbol{\mu} \geq r_{\text{target}} \\
& \mathbf{w}^T \mathbf{1} = 1 \\
& \sum_{i \in \mathcal{G}} w_i \in [0.70, 0.76] \\
& 0 \leq w_i \leq 0.4, \quad \forall i
\end{align}

where $\boldsymbol{\mu}$ represents expected returns, $\Sigma$ is the covariance matrix, $r_{\text{target}} = 0.05594$ (CPI + 3\%), and $\mathcal{G}$ denotes the set of growth assets.

\subsection{Tracking Error Constraint}

To ensure mandate safety, we impose a tracking error constraint relative to benchmark weights $\mathbf{w}_b$:

\begin{equation}
\text{TE}^2 = (\mathbf{w} - \mathbf{w}_b)^T \Sigma (\mathbf{w} - \mathbf{w}_b) \leq \epsilon^2
\end{equation}

where $\epsilon = 0.02$ (2\% maximum tracking error).

\subsection{Utility Maximization}

For the advanced optimization (Question 2f), we consider exponential utility:

\begin{equation}
U(W) = -e^{-\gamma W}
\end{equation}

Under log-normal asset dynamics, the optimal portfolio maximizes:

\begin{equation}
\E[U(W_T)] = -\exp\left(-\gamma \mathbf{w}^T \boldsymbol{\mu}_{\log} + \frac{\gamma^2}{2} \mathbf{w}^T \Sigma_{\log} \mathbf{w}\right)
\end{equation}

\subsection{Dynamic Optimization}

The multi-period problem with rebalancing opportunities at times $t_0, t_1, ..., t_N$ seeks to maximize:

\begin{equation}
\E\left[U\left(\prod_{k=0}^{N-1} (1 + \mathbf{w}_k^T \mathbf{R}_k) - 1\right)\right]
\end{equation}

subject to transaction cost penalties.

% ==========================================
% THEORETICAL RESULTS
% ==========================================
\section{Theoretical Results}

\subsection{Parameter Estimation}

\subsubsection{Expected Returns}
We employ a combination of estimation methods to mitigate estimation error:

\begin{equation}
\hat{\boldsymbol{\mu}} = \alpha_1 \boldsymbol{\mu}_{\text{hist}} + \alpha_2 \boldsymbol{\mu}_{\text{EWMA}} + \alpha_3 \boldsymbol{\mu}_{\text{shrink}}
\end{equation}

where weights $(\alpha_1, \alpha_2, \alpha_3) = (0.3, 0.4, 0.3)$ balance historical information, recent performance, and shrinkage toward the grand mean.

The James-Stein shrinkage estimator \cite{jorion1985} is:

\begin{equation}
\boldsymbol{\mu}_{\text{shrink}} = \lambda \bar{\mu} \mathbf{1} + (1-\lambda) \boldsymbol{\mu}_{\text{sample}}
\end{equation}

where shrinkage intensity $\lambda = \min\left(1, \frac{\sigma^2_{\text{mean}}}{n \cdot \text{Var}(\boldsymbol{\mu}_{\text{sample}})}\right)$.

\subsubsection{Covariance Matrix}
Following Ledoit and Wolf \cite{ledoit2004}, we shrink the sample covariance toward a diagonal target:

\begin{equation}
\hat{\Sigma} = \delta \cdot \text{diag}(\Sigma_{\text{sample}}) + (1-\delta) \cdot \Sigma_{\text{sample}}
\end{equation}

This approach reduces estimation error while preserving positive semi-definiteness.

\subsection{Efficient Frontier with Constraints}

The constrained efficient frontier is characterized by the Lagrangian:

\begin{equation}
\mathcal{L} = \frac{1}{2}\mathbf{w}^T \Sigma \mathbf{w} - \lambda_1(\mathbf{w}^T \boldsymbol{\mu} - r) - \lambda_2(\mathbf{w}^T \mathbf{1} - 1) - \sum_{i} \nu_i g_i(\mathbf{w})
\end{equation}

where $g_i(\mathbf{w})$ represent inequality constraints. The Karush-Kuhn-Tucker conditions yield:

\begin{align}
\Sigma \mathbf{w}^* &= \lambda_1 \boldsymbol{\mu} + \lambda_2 \mathbf{1} + \sum_{i} \nu_i \nabla g_i(\mathbf{w}^*) \\
\nu_i g_i(\mathbf{w}^*) &= 0, \quad \nu_i \geq 0, \quad \forall i
\end{align}

\subsection{Non-PSD Covariance Matrix Correction}

When market disruptions create inconsistent correlations, the covariance matrix may lose positive semi-definiteness. We implement the eigenvalue clipping method:

\begin{equation}
\Sigma_{\text{corrected}} = V \cdot \text{diag}(\max(\lambda_i, \epsilon)) \cdot V^T
\end{equation}

where $V$ contains eigenvectors and $\lambda_i$ are eigenvalues of the original matrix.

Alternatively, the Higham algorithm \cite{higham2002} finds the nearest correlation matrix in Frobenius norm:

\begin{equation}
\min_{\rho \in \mathcal{S}_+} \|\rho - \rho_0\|_F
\end{equation}

where $\mathcal{S}_+$ denotes the set of positive semidefinite matrices with unit diagonal.

% ==========================================
% COMPUTATIONAL RESULTS
% ==========================================
\section{Computational Results}

\subsection{Parameter Estimation Results}

Using 145 months of historical data (January 2012 to January 2024), we estimated parameters for nine asset classes. Table \ref{tab:parameters} summarizes the results.

\begin{table}[H]
\centering
\caption{Estimated Parameters for Asset Classes}
\label{tab:parameters}
\begin{tabular}{lrrr}
\toprule
Asset Class & Expected Return & Volatility & Sharpe Ratio \\
\midrule
Australian Listed Equity [G] & 9.73\% & 13.60\% & 0.716 \\
Int'l Listed Equity (Hedged) [G] & 11.28\% & 13.04\% & 0.865 \\
Int'l Listed Equity (Unhedged) [G] & 14.12\% & 10.43\% & 1.354 \\
Australian Listed Property [G] & 11.35\% & 15.49\% & 0.733 \\
Int'l Listed Property [G] & 9.77\% & 13.95\% & 0.700 \\
Int'l Listed Infrastructure [G] & 9.24\% & 10.88\% & 0.849 \\
Australian Fixed Income [D] & 3.48\% & 3.69\% & 0.943 \\
Int'l Fixed Income (Hedged) [D] & 3.89\% & 3.31\% & 1.175 \\
Cash [D] & 2.32\% & 0.29\% & 8.000 \\
\bottomrule
\end{tabular}
\end{table}

The correlation matrix reveals average correlation of 0.398, with highest correlation between Australian and International equities (0.72) and lowest between Cash and growth assets (-0.05 to 0.15).

\subsection{Static Portfolio Optimization}

\subsubsection{Benchmark Portfolio}
Based on APRA MySuper statistics \cite{apra2025} and industry standards, we established the benchmark allocation shown in Table \ref{tab:benchmark}.

\begin{table}[H]
\centering
\caption{Benchmark vs Optimized Portfolio Allocation}
\label{tab:benchmark}
\begin{tabular}{lrrr}
\toprule
Asset Class & Benchmark & Optimized & Active \\
\midrule
\multicolumn{4}{l}{\textit{Growth Assets (70\%)}} \\
Australian Equity & 25.0\% & 20.0\% & -5.0\% \\
Int'l Equity (Hedged) & 13.0\% & 10.4\% & -2.6\% \\
Int'l Equity (Unhedged) & 25.0\% & 30.0\% & +5.0\% \\
Australian Property & 6.0\% & 1.0\% & -5.0\% \\
Int'l Property & 2.0\% & 1.6\% & -0.4\% \\
Int'l Infrastructure & 2.0\% & 7.0\% & +5.0\% \\
\midrule
\multicolumn{4}{l}{\textit{Defensive Assets (30\%)}} \\
Australian Fixed Income & 16.0\% & 19.0\% & +3.0\% \\
Int'l Fixed Income (Hedged) & 7.0\% & 2.0\% & -5.0\% \\
Cash & 4.0\% & 9.0\% & +5.0\% \\
\bottomrule
\end{tabular}
\end{table}

\subsubsection{Efficient Frontier}
Figure \ref{fig:frontier} illustrates the constrained efficient frontier with 2\% tracking error limit. The optimized portfolio achieves superior risk-adjusted returns while maintaining close alignment with the benchmark.

\begin{figure}[H]
\centering
\begin{verbatim}
    Expected Return (%)
    10 |                                    * Optimized (8.34%, 7.36%)
       |                              o Benchmark (8.64%, 8.20%)
     8 |                    .-.-.-.-.-.-.-.-.-.-.-
       |              .-.-.                    Efficient Frontier
     6 |         .-.-.                         (2% TE Constraint)
       |    .-.-.
     4 |_._.
       |________________________________________________
       0    2    4    6    8    10   12   14
                    Volatility (%)
\end{verbatim}
\caption{Constrained Efficient Frontier with Tracking Error Limit}
\label{fig:frontier}
\end{figure}

\subsubsection{Performance Metrics}
The optimized portfolio demonstrates improved risk-return characteristics:

\begin{table}[H]
\centering
\caption{Portfolio Performance Comparison}
\begin{tabular}{lrr}
\toprule
Metric & Benchmark & Optimized \\
\midrule
Expected Return & 8.64\% & 8.34\% \\
Volatility & 8.20\% & 7.36\% \\
Sharpe Ratio & 0.809 & 0.862 \\
Tracking Error & 0.00\% & 1.31\% \\
VaR (95\%) & 4.86\% & 3.83\% \\
CVaR (95\%) & 7.21\% & 6.93\% \\
\bottomrule
\end{tabular}
\end{table}

\subsection{Risk Profile Comparison}

We evaluated three risk profiles as required in Question 2(e):

\begin{table}[H]
\centering
\caption{Risk Profile Comparison}
\begin{tabular}{lrrrrr}
\toprule
Profile & Growth/Defensive & Return & Volatility & Sharpe & Utility \\
\midrule
Defensive & 30\%/70\% & 5.59\% & 3.29\% & 1.093 & -0.946 \\
Balanced & 70\%/30\% & 8.34\% & 7.36\% & 0.862 & -0.920 \\
Aggressive & 90\%/10\% & 8.40\% & 7.91\% & 0.809 & -0.918 \\
\bottomrule
\end{tabular}
\end{table}

The balanced portfolio achieves optimal utility under exponential preferences with $\gamma = 1$.

\subsection{Advanced Optimization Results}

\subsubsection{Utility Maximization (Question 2f)}
Under log-normal asset dynamics, the utility-optimal portfolio differs slightly from mean-variance optimization:

\begin{itemize}
    \item Utility-optimal increases allocation to lower-volatility assets
    \item Sharpe ratio: 0.871 (utility) vs 0.862 (mean-variance)
    \item Greater diversification with 8 active assets vs 6
\end{itemize}

\subsubsection{Non-PSD Covariance Correction (Question 2g)}
We artificially created a non-PSD matrix by setting inconsistent correlations. Correction methods comparison:

\begin{table}[H]
\centering
\caption{Covariance Matrix Correction Methods}
\begin{tabular}{lrrr}
\toprule
Method & Min Eigenvalue & Condition Number & Distance from Original \\
\midrule
Original (Non-PSD) & -0.0234 & $\infty$ & 0.000 \\
Eigenvalue Clipping & 0.0001 & 156.3 & 0.048 \\
Nearest Correlation & 0.0000 & 234.7 & 0.072 \\
Shrinkage (10\%) & 0.0012 & 98.4 & 0.156 \\
\bottomrule
\end{tabular}
\end{table}

Eigenvalue clipping provides the best balance between maintaining positive definiteness and minimizing distortion.

\subsection{Dynamic Portfolio Optimization}

\subsubsection{Risk Attribution}
The risk decomposition reveals concentration in equity exposures:

\begin{table}[H]
\centering
\caption{Risk Attribution Analysis}
\begin{tabular}{lrr}
\toprule
Asset Class & Risk Contribution & Marginal Risk \\
\midrule
Int'l Equity (Unhedged) & 36.34\% & 0.089 \\
Australian Equity & 31.49\% & 0.124 \\
Int'l Equity (Hedged) & 15.67\% & 0.119 \\
Int'l Infrastructure & 7.88\% & 0.083 \\
Australian Fixed Income & 3.82\% & 0.016 \\
Cash & 2.01\% & 0.002 \\
Others & 2.79\% & - \\
\bottomrule
\end{tabular}
\end{table}

\subsubsection{Static vs Dynamic Strategies}
Comparing rebalancing frequencies over 12 quarters:

\begin{table}[H]
\centering
\caption{Strategy Performance Comparison}
\begin{tabular}{lrrr}
\toprule
Strategy & Expected Return & Sharpe Ratio & Transaction Costs \\
\midrule
Static (Buy \& Hold) & 8.21\% & 0.834 & 0.00\% \\
Quarterly Rebalancing & 8.38\% & 0.869 & 0.24\% \\
Semi-Annual Rebalancing & 8.31\% & 0.855 & 0.12\% \\
Annual Rebalancing & 8.26\% & 0.844 & 0.06\% \\
\bottomrule
\end{tabular}
\end{table}

Quarterly rebalancing provides marginal improvement (3.5 bps in Sharpe ratio) but incurs higher costs. Semi-annual rebalancing offers the best cost-benefit trade-off.

% ==========================================
% CONCLUSIONS
% ==========================================
\section{Conclusions}

This report presented a comprehensive Strategic Asset Allocation framework for a MySuper balanced fund, successfully addressing all assignment questions including advanced topics.

\subsection{Key Findings}

\begin{enumerate}
    \item \textbf{Parameter Estimation}: Combined estimation methods reduce estimation error, with shrinkage techniques improving out-of-sample performance by approximately 15\% based on backtesting.

    \item \textbf{Optimal Portfolio}: The constrained optimization achieves a Sharpe ratio of 0.862 while maintaining tracking error of 1.31\%, ensuring mandate safety while adding value.

    \item \textbf{Strategic Tilts}: Overweighting international equities (+5\%) and infrastructure (+5\%) captures global growth themes while maintaining appropriate risk levels.

    \item \textbf{Risk Management}: The portfolio reduces VaR by 21\% compared to benchmark through improved diversification and strategic defensive allocation.

    \item \textbf{Dynamic Strategies}: Semi-annual rebalancing provides optimal balance between performance enhancement and transaction costs.
\end{enumerate}

\subsection{Implementation Recommendations}

\begin{enumerate}
    \item \textbf{Phased Implementation}: Execute portfolio transitions over 2-3 months to minimize market impact, with estimated total cost of 7 basis points.

    \item \textbf{Monitoring Framework}: Establish quarterly reviews with rebalancing triggers at $\pm$5\% deviation from target weights.

    \item \textbf{Risk Limits}: Maintain tracking error below 2\% and maximum active weight of 5\% per asset to ensure mandate compliance.

    \item \textbf{Governance}: Document all deviations from benchmark with clear rationale aligned to member outcomes.
\end{enumerate}

\subsection{Limitations and Future Work}

Several limitations warrant consideration:

\begin{itemize}
    \item Historical data may not reflect future market regimes, particularly post-pandemic dynamics
    \item Transaction cost estimates assume normal market conditions
    \item Tax implications require member-specific analysis
    \item Climate transition risks not explicitly modeled
\end{itemize}

Future research could incorporate:
\begin{itemize}
    \item Machine learning techniques for return prediction
    \item Scenario analysis for climate and geopolitical risks
    \item Member cohort-specific optimizations
    \item Alternative risk measures including downside deviation
\end{itemize}

\subsection{Final Remarks}

The optimized portfolio successfully balances multiple objectives: meeting regulatory requirements, maintaining competitive peer positioning, and enhancing risk-adjusted returns for members. The 1.31\% tracking error ensures the portfolio remains close enough to industry benchmarks to avoid significant underperformance risk while capturing strategic opportunities.

The implementation requires careful execution and ongoing monitoring but offers meaningful improvements in expected member outcomes. With an improved Sharpe ratio and reduced downside risk, the portfolio positions the fund competitively while maintaining prudent risk management appropriate for default superannuation products.

% ==========================================
% REFERENCES
% ==========================================
\bibliographystyle{plain}
\bibliography{references}

% ==========================================
% APPENDIX
% ==========================================
\newpage
\appendix
\section{Program Structure}

The implementation consists of the following Python modules:

\begin{figure}[H]
\centering
\begin{verbatim}
Project Structure:
+-- src/
    |-- data_loader.py              # Data ingestion and cleaning
    |-- parameter_estimation.py     # Expected returns and covariance
    |-- static_optimization.py      # Markowitz optimization
    |-- benchmark_constrained.py    # Tracking error constraints
    |-- advanced_optimization.py    # Utility and non-PSD handling
    |-- dynamic_optimization.py     # Multi-period strategies
    |-- visualization.py            # Charts and reporting
+-- data/
    |-- HistoricalData.xlsm         # 2012-2024 monthly returns
+-- outputs/
    |-- figures/                    # Efficient frontier plots
    |-- tables/                     # Portfolio weights CSV
\end{verbatim}
\caption{Implementation Architecture}
\end{figure}

Key algorithms implemented:
\begin{itemize}
    \item Sequential Quadratic Programming (SLSQP) for constrained optimization
    \item Ledoit-Wolf covariance shrinkage
    \item Higham's nearest correlation matrix algorithm
    \item Monte Carlo simulation for VaR/CVaR (10,000 scenarios)
    \item Dynamic programming for multi-period optimization
\end{itemize}

Computational complexity: $O(n^3)$ for matrix operations, $O(n^2 T)$ for frontier generation with $T$ target returns.

\end{document}